\documentclass[11pt,a4paper]{report} 

% Für doppelseitigen Ausdruck (nur bei > 60 Seiten sinnvoll)
% \usepackage{ifthen}
% \setboolean{@twoside}{true}
% \setboolean{@openright}{true} 
%Wahrscheinlich eher nicht

\include{preamble} % alle Pakete und Einstellungen

% Hier anpassen 
\newcommand{\welchethesis}{Bachelor}
\newcommand{\thesisofwas}{of Science}
\newcommand{\studiengang}{Technische Informatik}
\newcommand{\titel}{Arbeitstitel: Anforderungen und Testen}
\newcommand{\kurztitel}{Arbeitstitel: Anforderungen und Testen}
\newcommand{\autor}{Florian Brunotte}
\newcommand{\datum}{30.01.2021} % Abgabedatum
\newcommand{\ort}{Mannheim}
\newcommand{\referent}{Prof.\ Dr.\ Ing. Mark Hastenteufel}
\newcommand{\korreferent}{Prof.\ Dr.\ Martin Damm}

\begin{document}

\include{vorspann} % Titelseite, Erklärungen, etc.

\begin{abstract} 
Ziel dieser Bachelorarbeit war die Erstellung einer leichtgewichtigen Software mit der Requirements und Testcases aufgenommen werden können und durch Testruns validiert werden. (Unterschied Validierung und Verifikation?) Eingesetzt wird diese Software im Rahmen von kleinen Semesterprojekten an der Hochschule Mannheim. Geschrieben wurde sie in als Webapplikation in Python und Django während die Visualiisierung und das User Interface durch HTML und CSS erstellt wurde.
\end{abstract}

\tableofcontents %Inhaltsverzeichnis wird erstellt
\chapter{Motivation - Warum wurde diese Software geschrieben?} \label{chap:motivation}
Es ist vorzuziehen, dass...

Motivation:
Die Zielgruppe für diese Arbeit und für diese Software sind die Hochschulangehörigen, die innerhalb des Studiums Anforderungen und Tests verwalten müssen. 

Es soll eine leichtgewichtiges Softwaretool entwickelt werden zum Anforderungs- und Testmanagement. Dabei sollte man sich auf die Kernfunktionalitäten beschränken, damit man sich auf das Wesentliche konzentrieren kann, dem Schreiben von Anforderungen und dem Ausführen von Tests. Andere Softwaretools, die am Markt verfügbar sind, sind für den Einsatz in der Lehre nicht geeignet, da sie zu komplex sind und auch Geld kosten können. Sollte man nicht bezahlen, werden bestimmte Funktionen gesperrt. Insgesamt sind sie zu schwergewichtig und man verliert das eben beschriebene wesentliche aus dem Blick. Eine intuitive und simple Software, wie ,,Anforderungen und Testen (WIP)'' soll hierbei Abhilfe schaffen.

Hier werden dann noch manche Tools aufgezählt. Vielleicht mit Bilder die Limitierung und die Überfülle zeigen.

\chapter{Anforderungsanalyse - Welche Anforderungen hat das Projekt?} \label{chap:Anforderungen}

Zuerst hat Herr Hastenteufel ein Dokument mit Anforderungen gesendet. Nach Absprache per Mail wurden noch folgende Sachen geklärt:
\begin{itemize}
\item Die Kategorie für die Requirements werden erstmal als einfache Kategorien wie 1, 2, 3 realisiert. Später vielleicht Auswahl aus der Art der Requirement: Stakeholderanforderung, aber dazu nochmal in SOE reingucken

\item Die Namen Requirement, Testcase und Testrun werden benutzt, um alles einheitlich zu lassen

\item Da der Admin und der Professor als Nutzer sehr ähnlich sind, wurden die beiden Rollen mit ihren Anforderungen zusammengelegt als Professor 
\end{itemize}


Ein Klassendiagramm und ein Zustandsdiagramm wurden erstmal nicht gemacht, da sie nicht für sinnvoll gehalten wurden. Aber das wurde nochmal nachgefragt.

Es wurden ein Use Case Diagramm skizziert, mehrere Activity Diagramme, die die Use Cases verfeinern skizziert und die UIs wurden auch skizziert. Daneben wurde auch das ER-Modell schon in Draw.io gezeichnet und die Tabellen daraus wurden transformiert. Somit hat man für den Prototyp bereits ein Datenmodell mit dem man testen kann. Die Diagramme werden sich wahrscheinlich noch ändern im Verlauf des Projekts, aber für den Anfang sind sie in den Abbildungen 1, 2, 3... zu sehen. Das war der Start des Projekts.

Es wurden ebenfalls erste Skizzen zu den UIs vollzogen und abgesprochen mit dem Kunden, also Herrn Hastenteufel.
Die Rückmeldungen waren:




\chapter{Implementierung - Wie erstellt man eine Django-Applikation?} \label{chap:einf}
%So geht ein Zitat oben rechts
%das sollte auch hier stehen bleiben sonst sieht es nicht gut aus und der Text überlappt
\epigraphhead[70]{\epigraph{Documentation is like sex: 
when it is good, it is very, very good; and when it is bad, 
it is better than nothing.}{\textit{Dick Brandon}}}

mo: Warum sollte man Django verwenden ist in der Tabelle tab dargestellt:
mo: Geschichte von Django und die Popularität
mo: den generellen Aufbau und Ablauf eines HTTP Request kann man in der Abbildung abb sehen. In Django Applikationen werden die verschiedenen Aufgaben von verschiedenen Dateien erledigt. In URLS.py werden die Views zu den HTML Seiten gelinkt. Sie verknüpfen also die spezifischen URL Patterns mit den gewünschten View Funktionen. In den Views werden noch die HTML Seiten mit Daten versehen aus den Modellen, die man zuvor definiert hat. Eine View verarbeitet die HTTP requests und sendet basierend auf diesen die HTTP responses. Hierbei ist die render Funktion entscheidet. Models sind Python objects und stellen so Funktionen zur Verfügung zum erstellen, ändern und löschen der Daten, aber auch um die Daten in die Datenbank zu speichern mit Querys. Die Templates können dynamisch HTML Seiten erzeugen. Sie haben noch Platzhalter für die richtigen Inhalte. Es müssen aber keine HTML Dateien sein. Django nennt diese Methode Model View Template und das kann man vergleichen mit dem Model View Controller Architektur.

mo: Das Model View Controller Architektur kann auf folgenden Weg beschrieben werden... und mit der Abbildung abb auch erklärt werden.

mo: Die Kommunikation mit der Datenbank wird von Django erledigt, das heißt die beiden Sachen sind entkoppelt voneinander. Man kann so die Models einfach definieren und dann die Datenbank auswählen. 

mo: Python unterstützt die Objektorientierte Programmierung bzw. ist eine objektorienritere Programmiersprache. 

mo: Django hat von sich aus einen Development Webserver dabei, den man benutzen kann um seine Anwendung zu testen. 

mo: zur Instalation wurde in PyCharm ein neues Projekt angelegt und dann in dem virtual environment kann mit pip die verschiedenen Python Erweiterungen runtergeladen werden. Man muss sich hierbei entscheiden ob man es global installiert, dann kann man aber nur von jeder Sache nur eine Version gleichzeitig da sein. Bei einem virtual Environmnet kann man dann in einer eigenen Version verschiedene Versionen unterbringen. Die Sachen gelten dann auch nur in dem Environment. Die hier verwndeten Versionen von Python ist 3.7.2, von Django ist 3.1.2 und von PIP ist 1. Die verwendete Datenbank ist Postgres und die Version davon ist 3.7. In der Tabelle sind die verwendeten Versionen und Software mit Download Link angegeben. Virtualwrapper mit virtualvenv ist auch verwendet worden. In Pycharm wird bei einem neuen Projekt auch ein neues venv gemacht. Zum venv gibt es die verschidenen Befehle workon, deactivate, remove. 

%Quellen sind hier das Django und das Mozilla Tutorial
Mit PyCharm ein neues Projekt starten. Dabei wird ein neues virtual Enviroment gebildet in dem man alles installieren kann, das hat aber keine Auswirkungen auf andere Installationen, wie die Hauptinstallation von Python. Neuer Ordner wurde gebildet in dem Bachelorarbeit Ordner für den Prototyp. Als nächstes sollte man Django installieren. Im Terminal von PyCharm kann alles gemacht werden, wie Django installieren. Mit Pip install django kann man dann django installieren. Ebenfalls sollte hier bereits die Datenbank wie Postgres installiert werden. Nach anlegen des Projekts kann die Datenbank auch schon registriert werden. Wichtig hierbei sind das Passwort für den Benutzer und der Benutzername neben dem Namen der Datenbank.

Dann einfach das Tutorial machen von Django Documentation.

Django entstand in einem Nachrichtenumfeld als Einführung in Django

Man startet ein neues Django Project mit 
\verb|django-admin startproject anforderungenundtesten|
, dabei sollte der Name nicht wie django oder test heißen, das könnte zu Problemen führen \verb|cd ./anforderungenundtesten| um in den richtigen Ordner zu kommen, der die \verb|manage.py|  Datei enthält. Durch diesen Schritt wurde in dem ausgewählten Ordner ein Django Projekt erstellt und dazu notwendiger Code autogeneriert. In der Tabelle ~\ref{tab:AutoCode} können die bis jetzt generierten Dateien und Ordner gesehen werden.


\begin{table}
\centering
\begin{tabular}{|p{0.3\textwidth}|p{0.7\textwidth}|}
\hline
\multicolumn{1}{|c|}{\textbf{Datei}} & 
\multicolumn{1}{|c|}{\textbf{Inhalt}} \\\hline
/anforderungenundtesten & Der äußere Ordner ist ein Container für das Projekt, der Name ist nicht wichtig und kann beliebig unbenannt werden\\\hline
manage.py     & In dieser Datei sind die Command Line Befehle enthalten, wie zum Beispiel runserver \\\hline
anforderungenundtesten/    & der innere Ordner ist der eigentliche Python Package mit diesem Namen werden alle Dateien importiert, wie zum Beispiel anforderungenundtesten.urls \\\hline   
\end{tabular}
\caption{Die erstellten Dateien und Ordner nach dem django-admin startproject anforderungenundtesten Befehl}
\label{tab:AutoCode}
\end{table}




über python manage.py runserver kann der Developerserver aufgerufen werden und man sieht, dass die Installation geklappt hat. Man kann hierbei auch die IP Adresse so ändern, dass alle Geräte im gleichen Netzwerk Zugriff haben. Weitergehend kann unter PyCharm die Run Configuration so angepasst werden, dass bei einem Run der runserver Befehl aufgerufen wird. Über STR C kann der Server wieder beendet werden.


Es gibt dabei einen Unterschied zwischen einem Projekt und einer App in Django, so kann ein Projekt mehrere Apps enthalten und eine App kann in verschiedenen Projekten enthalten sein. Man sollte auf die Wiederverwendbarkeit achten. Eine App ist auch etwas, dass etwas macht, wie ein Blog oder eine Datenbank auslesen und Daten darstellen. Ein Projekt ist dann eine Sammlung von Apps und Einstellungen für eine explizite Webseite.

Über den Befehl \verb|python manage.py startapp aut| wird eine neue App angelegt und auch hier wird zuerst wieder Code autogeneriert. Dieser ist in der Tabelle ~\ref{tab:AutoCode2} dargestellt mit seinen Komponenten und den verschiedenen Funktionen.


\begin{table}
\centering
\begin{tabular}{|p{0.3\textwidth}|p{0.7\textwidth}|}
\hline
\multicolumn{1}{|c|}{\textbf{Datei}} & 
\multicolumn{1}{|c|}{\textbf{Inhalt}} \\\hline
/aut & Der Ordner beinhaltet die App und die nachfolgenden Dateien\\\hline
manage.py     & In dieser Datei sind die Command Line Befehle enthalten, wie zum Beispiel runserver \\\hline
anforderungenundtesten/    & der innere Ordner ist der eigentliche Python Package mit diesem Namen werden alle Dateien importiert, wie zum Beispiel anforderungenundtesten.urls \\\hline   
\end{tabular}
\caption{Die erstellten Dateien und Ordner nach dem django-admin startproject anforderungenundtesten Befehl}
\label{tab:AutoCode2}
\end{table}


Hierbei kam der Begriff Boiler Code auf, der Codeabschnitte beschreibt, die wenige Änderungen haben und an vielen Stellen auftauchen können. Beispiele sind in den Listings 1 und 2 dargestellt. Man sieht in dem 1. Lisitng ein Shebang für die Programmiersprache Python, damit wird dem Compiler klar das es sich hier um Python Code handelt. Im 2. Lisiting sieht man die Basis von einem HTML Dokument. Mit diesem Template können viele verschiedene Websites generiert werden, aber diese Tags sind notwendig. %Quelle Wikipedia --> Verbesserungswürdig

Die App muss unter den installierten Apps angemeldet werden

Danach sollten die URLs definiert werden und zusammengeführt werden. Es gibt eine url Datei für das Projekt und eine für die App. Von dem Projekt kann man auch die Grundseite umleiten lassen auf die Seite der App. Die verschiedenen Routen der Seiten sind in der Abbildung dargestellt. 
In dieser Datei kann man über den Path angeben welche View für welchen HTTP Request ausgeführt werden soll. Das soll also eine Verlinkung sein.

Die andere url Datei vom Hauptprogramm, also anforderungenundtesten, ist dafür zuständig den Hauptpfad zu deklarieren. In dieser Arbeit ist der Hauptpfad "`aut"'. Danach wird das url File von der App inkludiert mit \verb|include()| Daneben wird auch die admin Seite angegeben.

In den Settings kann auch noch die TIMEZONE angepasst werden. 

mo: die 2 anderen Einstellungen, die man beachten sollte sind der Secret Key und DEBUG. Mit dem Secret Key kann Django die Security Sachen machen, darum sollte man ihn geheim halten und schützen während der Produktionsphase. Mit dem DEBUG Keyowrd kann man mehr Informationen bei einem Fehler bekommen, wenn die App dann fertig ist, sollte er False sein.

Es müssen auch die Befehle \verb|makemigrations| und\verb|migrate| angewandt werden, da sie sämtliche Änderungen an den Datenbanken und den Modellen ausführen.  Mit \verb|migrate| werden alle notwendigen Datenbank Tabellen erstellt für die unter InstalledApps angegebenen Apps. In der Postgres Umgebung können diese Tabellen auch gesehen werden. 

Zuletzt wird mit runserver der Development Server erstellt und zum Laufen gebracht unter der Adresse kann man dann lokal die App testen oder man kann auch für alle Geräte im gleichen Netz die App verfügbar machen. Das ist nützlich, um die Webseiten auf mobilen Geräten oder generell Geräten mit verschiedenen Displaygrößen zu testen.

Die Datenbank, in diesem Fall Postgres muss auch eingerichtet werden, dazu gibt es in settings.py die Einstellungen zu Databases. Diese müssen verändert werden, damit das Django-Projekt mit einer Postgres Datenbank funktioniert. Es müssen der Name und das Passwort des Benutzers zum Anmelden benutzt werden, nicht das Masterpassword.  


Danach können die Modelle definiert werden. ORM von Python ist hierbei wichtig. Die Modelle werden aus den Tabellen, die wiederrum aus den ER-Modellen hergeleitet werden, erstellt. Bei Modellen hat Django das Dont Repeat Yourself Prinzip, welches besagt dass...

Die ER-Modelle sind in der Abbildung 1 dargestellt....weiter erklären. Die Modelle sind in Abbildung 1 dargestellt... auch weiter erklären. Zuerst werden einfache Modelle angelegt in Django mit Foreign Keys und Char Feldern. Der Primary Key ist definiert so wie er als Standard definiert wäre, aber das wurde überschrieben, um den einzelnen Schlüsseln die richtigen Namen zu geben. Ansonsten ist der Primary Key einfach ein AutoField und hat Integer Werte die bei 1 anfangen. Die Foreign KEys sind auch damit OK, dass sie keinen Wert am Anfang haben. 

Durch diese Beschreibung der Modelle kann Django die entsprechenden Datenbanken erstellen und für den Entwickler die API bereitstellen.

Nachdem man die Modelle geschrieben hat muss man das Modul noch für den Admin zur Bearbeitung freischalten bzw. registrieren. Der Admin oder Superuser muss auch erstellt werden. Unter installed Apps muss man dann aut noch eintragen.


Nachdem die Modelle erstellt/geschrieben wurden, kann man mit makemigrations und migrate die Tabellen/Relationen in die Datenbank einspielen. Mit makemigrations werden die migrations erstellt als Datei um noch mal drüber zu gucken mit den SQL Befehlen. Dabei konnte ich auch den Fehler mit dem Default DateTimeField ausbessern. Da die Migrationen nacheinander gemacht werden. Jetzt kann man bereits über den Admin die Daten einpflegen. Dazu ist zu sagen, dass das bis jetzt nur der Admin kann und die Admin Seite zwar Funktional ist, aber im Sinne von UI/Aussehen noch verbessert werden kann. Mit dem Befehl \verb|python manage.py sqlmigrate 0001| kann man die SQL Anweisungen zu dem entsprechenden Migrate sehen in der Konsole, z.b. zu 0001 initial. Der Befehl zeigt nur die Befehle an, die gemacht werden ohne diese tatsächlich zu machen.
Mit dem Befehl \verb|check| kann auch vorher gecheckt werden, ob es Probleme gibt mit den Migrationen ohne das diese schon gemacht werden.


Die Schritte beim Arbeiten mit der Datenbank und den Modellen ist wie folgt:
\begin{itemize}
\item Die Modelle anpassen in models.py
\item mit makemigrations die migrations für die Änderungen erstellen
\item mit migrate die migrations anwenden
\item optional: check um Fehler zu finden
\item optional: sqlmigrations um die SQL Befehle zu sehen
\end{itemize}


Der Admin muss zuerst angelegt werden über create superuser.
Und die Modelle müssen noch angemeldet werden, damit der Admin diese bearbeiten kann. Dann kann man über die Admin Seite bereits Daten einfügen, ändern und löschen. Im nächsten Schritt wird die Admin Seite aber angepasst, damit sie besser aussieht und auch neue Funktionalitäten hat. 

Mit dem runserver Befehl kann dann der Development Server aufgerufen werden, der die Seiten, die unter den Urls und den Views definiert sind anzeigt. Dazu kommt dann auch noch die admin Seite unter der die Modelle mit ersten Daten versorgt werden können und auch angezeigt werden können

mo: Mit der RedirectView kann man die GrundURL auf eine App lenken. Das macht hier Sinn, da es nur eine App geben wird aut.
mo: Damit die Seite auch noch static Files unterstützt, wie JS und CSS kann man auch noch eine Zeile hinzufügen.
mo: makemigration und migrate sind die wichtigen Befehle für die Datenbank 

mo: Django übernimmt die Kommunikation mit der Datenbank, darum muss man bei der Definition der Models nicht darauf achten.

mo: Bei den Modellen können eine Menge Attribute definiert werden. Zum Beispiel ein Hilfetext für den Nutzer in den HTML-Seiten. Wichtig ist auch die str Methode, damit die einzelnen Elemente etwas sinnvolles zurückgeben an dem man sie unterscheiden kann. Mit dem getabsolute url kann man den einzelnen Elementen sofort eine verlinkte Seite geben.

mo: f Strings könnten auch verwendet werden

mo: Jedes Model muss einzeln eingetragen werden in die Admin Seite. Wichtig hierbei ist, dass die Admin Seite ist automatisch erstellt worden. Nicht als eigenes verkaufen. Es können aber verschiedene Veränderungen gemacht werden zu den Admin Seiten. Damit kann doch ein eigenes Look and Feel hergestellt werden. 

mo: Die Templates sind dafür da dynamisch die HTML Seiten zu generieren und mit Daten zu füllen. In den Templates können dann Variablen angezeigt werden, die vorher aus der Datenbank generiert oder berechnet wurden. Man braucht hierzu die View und das template

mo: Wichtig ist hierbei die Render Funktion, die 3 Elemente weitergibt. Zum einen das request Objekt, ein HTML Template und eine context Variable, was ein Python Dictionary ist. 

mo: Es wird mal das Kapitel mit generic und detailed View ausprobiert an den Requirements. Es wurden einfache Seiten gemacht die die Elemente anzeigen und Details auch.

mo: Sessions wurden auch gemacht, es gibt einen Counter wie oft man die Webseite besucht hat

mo: Es gibt jetzt eigene Login und Logout Seiten, die noch erweitert werden sollten. Daneben kommen auch noch die Seiten für Passwörter, also zum resetten. Diese Sachen sind auch wieder von django vorgegeben. Die Funktion das Passwort zurückzusetzen musste simuliert werden, da Emails senden so noch nicht funktioniert. Dazu wird die gesendete Email mit dem Link in der Konsole geloggt. Es können auch Seiten so geschrieben werden, dass sie einen Login brauchen bevor man auf sie zugreifen kann.

mo: jetzt kommt die Hauptsache, die noch am wichtigsten ist: die Forms mit denen die Nutzer Daten eingeben können.

WICHTIG: Danach wird das Projekt einmal aufgeräumt, damit es übersichtlicher ist. Es wird dazu einen neuen Branch geben

Darunter fällt auch das JOIN von verschiedenen Tabellen, wie die einzelnen Tabellen miteinander verbunden wurde, kann man in den Abbildungen 1 sehen.
Mit so Pfeilen zwischen den Schlüsseln und so

Das UI wird aber später kommen, da man erstmal einen Prototyp haben möchte der die Grundlegenden Funktionen einer  Datenbank hat: Daten eintragen, löschen, ändern und anzeigen.

Das erste UI was man auch klicken kann wurde in Axure erstellt. Diese Dateien konnte man als HTML exportieren und auch CSS und Javascript war dabei. Der nächste große Schritt ist jetzt diese Dateien in das Django Projekt einzubinden. 

Bei der Admin Seite werden nur die Modelle angezeigt, die unter admin.py registriert sind. Darum müssen alle Modelle zuerst registriert werden


Im nächsten Schritt werden die Views kreiert, die man braucht. Django vergleicht welche View angezeigt werden soll durch die URL, darum mussten die Views auch erst verknüpft werden um jetzt benutzt werden zu können.
Bei diesen Views kann  man auch aus der Anfrage Variablen abfangen. Das heißt man kann zusätzliche Variablen angeben oder zum Beispiel ein Requirement im Detail angucken, wenn die entsprechende ID angegeben wurde in der URL. So kann man mit \verb|<int:>| etwas abfangen
Nachdem auf diesem Weg erste einfache Views geschrieben wurden, kann man jetzt das auf dem Weg machen, wie es auch in Django vorgesehen ist. Man erstellt templates, also html Dateien und tut sie in den entsprechenden Ordner. Der Pfad ist so gewählt, dass die templates wieder den Apps zugeordnet werden können. In den Templates wird dann auch Django Code verwendet mit den Klammern und den Befehlen wie if und so.

Zum einen muss man die richtigen Befehle benutzen um in den Views die richtigen Werte der Models zu bekommen. Das ist eine Qual. Danach sollte man die Views Sachen übergeben an das HTML und in die entsprechenden Felder die Sachen ausgeben. Dabei wird bei mir bis jetzt eine Liste mit Dictionarys übergeben. Das muss dann entsprechend mit 2 For Schleifen ausgegeben werden. Dann kann man so die Daten die bis jetzt in der Datenbank sind ausgeben in der HTML Datei, das war echt ein langer Weg. 

Um die richtigen Werte zu bekommen hat man zuerst die Elemente gefiltert nach dem richtigen Projekt. Das Projekt sollte später von irgendwo herkommen, wie zum Beispiel automatisch und unsichtbar für den Nutzer. Die Nutzer sind ja sowieso eingeteilt in die Gruppen und gehören nur zu einem Projekt. Danach werden alle Elemente dazu geholt, damit man auch auf die Attribute zugreifen kann wie Name oder so. Hier im Beispiel wird einfach alles ausgegeben. Mit den Forms die später kommen sollen kann man so vielleicht dann noch Sachen Filtern oder so. Damit sollte man dann auch endlich den anderen Weg beschreiten könne, dass man die Daten in die Datenbank schreibt ohne die Admin Seite benutzen zu müssen. Es ist bei alldem sehr wichtig auch die richtige view in den URLS verknüpft zu haben. 

Mit der render Funktion kann man eine Abkürzung machen. Man muss dann nicht mehr zuerst das Template laden, sondern kann es sofort in der render Funktion angeben.

Man kann den Apps in den url EInstellungen auch einen Namen geben, das ist vor allem nützlich, wenn man mehrere Apps hat und man aber in den Templates auf eine bestimmte zugreifen möchte. In diesem Beispiel ist der appname dann aut und man kann so auf die views zugreifen mit aut:requirement. Auf diese Weise können jetzt alle Links zu den verschiedenen Django Templates, also HTML Dateien geschreiben werden und man kann die Seite bereits navigieren.

Das Dashboard und die entsprechenden Seiten sind in der Abbildung abb angezeigt.

Die HTML Siten wurden mit einer Testversion von Axure erstellt. Mit der Seite kann man HTML Seiten erstellen und bereits ausprobieren. Bei der Integration muss daraug geachtet werden, dass die ganzen Dateien, also auch die CSS Dateien in den richtigen Ordnern sind. Die Links in den Dateien müssen noch angepasst werden an die static Variable. Die ganzen ressourcen, die mit src anfangen und die href müssen jetzt noch /static/ davor stehen haben, da die Sachen in diesem Ordner zu finden sind.

\chapter{\LaTeX-Sachen} \label{chap:latex}
\section{Latex Sachen als Section}

%So geht eine Aufzählung mit Punkten
Die Ziele des Templates sind wie folgt:
\begin{itemize}
\item Beispiele der typischen Verwendung von \LaTeX\ und dessen Erweiterungen 
  geben, die viele im Rahmen von Abschlussarbeiten üblichen Anforderungen 
  abdeckt.
\item Nahe am \LaTeX-Standard halten mit wenigen weit verbreiteten 
  Erweiterungen, um problemlosen Einsatz und Erweiterbarkeit sicher zu stellen.
\item Die Einhaltung der Formalien an der Hochschule~Mannheim in der
  Fakultät Informationstechnik vereinfachen.
\end{itemize}

%so kann man Sachen durchstreichen
\falsch{durchgestrichen} angezeigt. 


%so geht referenzieren, also mit der Tilde vorne ist wiochtig sonst gibt es komische Seitenumbrüche
Nehmen Sie Kapitel 1 nicht mit dazu. Schreiben Sie Inhalte und keine
Leerphrasen. Verwenden Sie nicht das "`nächste"' oder "`folgende"' Kapitel
sondern immer als Zahl das wievielte Kapitel. 
Verlassen Sie sich auf \LaTeX\ und nummerieren Sie nie selbst sondern
referenzieren Sie. Jedes Kapitel außer das Erste muss vorkommen. 
In Kapitel~\ref{chap:latex} führen wir in die \LaTeX-Umgebung kurz ein und 
geben eine Übersicht über die Tools, die notwendig sind ein Dokument zu 
erstellen.
In Kapitel~\ref{chap:layout} stellen wir das Layout sowie einige Idiome 
zum Textsatz mit \LaTeX\ vor. 
In Kapitel~\ref{chap:bilder} besprechen wir das Einbinden und Erstellen 
von Fließobjekten wie Bilder, Tabellen und Listings.
Hinweise zum Schreibstil, mathematischem Formelsatz und zur Literatur sind 
in Kapitel~\ref{chap:stil} gesammelt.
Abschließend fassen wir in Kapitel~\ref{chap:fazit} die Vorteile und Features,
hervorzuheben sind die gute Qualität und Satz,
von \LaTeX\ für Ihre Abschlussarbeit noch einmal zusammen.


%So geht eine Tabelle
in Tabelle~\ref{tab:disteditplattform}.
\begin{table}
\centering
\begin{tabular}{|l||l|l|}
\hline
\multicolumn{1}{|c|}{\textbf{Plattform}} & 
\multicolumn{1}{|c|}{\textbf{\LaTeX-Distribution}} & 
\multicolumn{1}{|c|}{\textbf{Editor}} \\\hline\hline
Linux/Unix & TeX Live         & Texmaker, Emacs \\\hline
MacOSX     & TeX Live, MacTex & Texmaker, TeXShop \\\hline
Windows    & TeX Live, MiKTeX & Texmaker, TeXstudio \\\hline   
\end{tabular}
\caption{\LaTeX-Distributionen und Editor je Plattform}
\label{tab:disteditplattform}
\end{table}

%Damit das so anders aussieht, zum Beispiel für Befehle wie makemigrations
(zum Beispiel \verb|thesis.tex|)



%So kann man andere Schriften machen
\noindent\parbox[t]{\textwidth}{
\fontfamily{ptm}\fontsize{11}{13pt}\selectfont \ 
(Times New Roman) When Apollo Mission Astronaut Neil Armstrong first 
walked on the moon, he not only gave his famous ``one small step 
for man, one giant leap for mankind'' statement but followed it by 
several remarks, usual communication traffic between him, the other
astronauts and Mission Control. 
Just before he re-entered the lander, however, he made this 
remark \textit{Good luck Mr. Gorsky}.
}

\noindent\parbox[t]{\textwidth}{
\fontfamily{phv}\fontsize{11}{13pt}\selectfont \ 
(Helvetica) Many people at NASA thought it was a casual remark concerning 
some rival Soviet Cosmonaut. 
However, upon checking, there was no Gorsky in either the
Russian or American space programs. 
Over the years many people questioned Armstrong as to what 
the \textit{Good luck Mr. Gorsky} statement meant, but Armstrong
always just smiled.
On July 5, 1995 in Tampa Bay FL, while answering questions following 
a speech, a reporter brought up the 26 year old question to Armstrong. 
This time he finally responded. Mr. Gorsky had finally died and so 
Neil Armstrong felt he could answer the question.
}

\noindent\parbox[t]{\textwidth}{
\fontfamily{pplx}\fontsize{11}{13pt}\selectfont \ 
(Palatino) When he was a kid, he was playing baseball with a friend 
in the backyard. His friend hit a fly ball, which landed in the front 
of his neighbor's bedroom windows. 
His neighbors were Mr. \& Mrs. Gorsky.
As he leaned down to pick up the ball, young Armstrong heard 
Mrs. Gorsky shouting at Mr. Gorsky. 
\textit{Oral sex! You want oral sex?! You'll get oral sex when the 
  kid next door walks on the moon!}
}

%So gehen Fußnoten
Machen Sie Fußnoten\footnote{Das ist eine Fußnote.} immer
ohne einleitendes Leerzeichen und innerhalb des Satzes,
also nie nach einem Punkt. 
Fußnoten sind ganze Sätze mit Satzzeichen.
Fußnoten sind Inhalte, die nicht für das Verständnis 
notwendig sind\footnote{Fußnoten haben übrigens 
nichts mit Noten oder Musik zu tun.}. 
Juristen verwenden Fußnoten zur Quellenangabe. Wir
sind keine Juristen und distanzieren uns 
(nicht nur) von dieser Praxis deutlich.
Setzen Sie Fußnoten sehr sehr sparsam ein.

%so geht das Referenzieren auch auf Seiten
Referenzieren Sie innerhalb des Dokuments, zum Beispiel
auf das Kapitel~\ref{chap:bilder} in dem es unter anderem
um Bilder geht und das auf Seite~\pageref{chap:bilder}
los geht, mit \verb|\ref| (meistens) oder 
\verb|pageref| (sehr selten). 
Verwenden Sie vor dem Befehl zum Referenzieren immer
ein \verb|~|. Das ist ein nicht umbrechbares Leerzeichen
und \falsch{Kapitel \\ 1}, also der 
Zeilenumbruch vor der Nummerierung, wird vermieden.

\falsch{Es macht keinen Sinn aus irgendwelchen Gr"unden}\newline
\falsch{erscheinen sie noch so sinnvoll}\\ 
\falsch{Zeilenbr"uche im Flie"stext einzuf"uhren.}\\
\falsch{Sie}\\
\falsch{wollen eigentlich etwas anderes.}

%so kann man Sachen kursiv machen
%Abkürzungsverzeichnis vielleicht wieder nehmen
Sie sollten Abkürzungen (AKÜ) bei ersten Vorkommen definieren.
Schreiben Sie das Wort zuerst aus und dann die Abkürzung in 
Klammern. 
Danach können Sie die AKÜ verwenden. 
Meistens sollten Sie jedoch auf Abkürzungen verzichten.
Schreiben Sie lieber
\textit{beispielsweise, zum Beispiel, und so weiter, beziehungsweise}
statt \textit{bspw., z.B., usw., bzw.}.

%So gehen Bindestriche
Setzen Sie die drei verschiedenen Bindestriche -, -- und --- richtig
ein. 
Der einfache Bindestrich - wird bei Worttrennungen, 
wie AKÜ-Fimmel, eingesetzt (im Quelltext mit \verb|-|).
Der etwas längere Streckenstrich -- wird bei Streckenangaben, wie
die Strecke Mannheim--Karlsruhe oder von 10:00--11:45 eingesetzt
(im Quelltext mit \verb|--|).
Der Gedankenstrich --- ist bei Einschüben --- wie zum Beispiel
hier --- einzusetzen (im Quelltext mit \verb|---|).
\falsch{Es ist auf keinen Fall ein Leerzeichen um einen Binde - strich 
oder einen Strecken -- strich und immer ein Leerzeichen
um einen Gedanken---strich}.


%noch eine Tabelle

\begin{table}[htbp] % htbp ~ here, top, bottom, page
\centering
\begin{tabular}{|r|c|l|l|}
\hline
\textbf{Name} & \textbf{Adresse} & \textbf{Wohnort} & \textbf{Telefon} \\ 
\hline\hline
Susi Sinnlos & Eichenstrasse 5 & 12345 Unterstadt & 24927749242 \\
Horst Kurz & Schnellweg 17 & 42420 Rapid & 999 \\\hline
Jochanaan Leuchtentrager & Hochstraße zu & 666 Hell & 1-800-33845\\\hline
\end{tabular}
\caption{Adressliste}
\label{tab:meinetab}
\end{table}

Die entsprechenden vordefinierten Umgebungen heißen 
\verb|table| für Tabellen und \verb|figure| für Abbildungen. 
Mit den optionalen Argumenten \verb|htbp|, das steht für
\textit{here, top, bottom, page}, geben Sie \LaTeX\ den 
Tipp, dass Sie am liebsten das Fließobjekt \textit{hier}
an dieser Stelle haben möchten. Wenn das nicht geht, dann
eben am \textit{Anfang} der Seite, und wenn das nicht geht (weil es
zum Beispiel ein Kapitelanfang ist) ans \textit{Ende} der Seite. 
Wenn das alles nicht klappt, dann halt auf eine Extra-\textit{Seite}.
Beherzigen Sie folgende Tipps zu Fließobjekten:
\begin{itemize}
\item Jede Tabelle, jedes Bild und jedes Listing ist ein Fließobjekt.
\item Zentrieren Sie Bilder und Tabellen.
\item Jedes Fließobjekt hat eine Bildunterschrift (Caption) mit
  einem Label und wird im Text passend referenziert.
\end{itemize}
Schreiben Sie nie, \falsch{wie man unten in der Tabelle sehen kann},
da Sie nie wissen und auch nicht wissen sollen, ob die Tabelle 
wirklich \glqq weiter unten\grqq\ ist. 
Verwenden Sie statt relativer Positionsangaben Referenzen mit Zahlen,
die Sie durch das Label erhalten, wie zum Beispiel 
"`wie sie in Tabelle~\ref{tab:meinetab} sehen können"'.
Verwenden Sie kurze und prägnante Bildunterschriften, die 
nicht länger als eine Zeile lang sind. 
Alles was mehr als eine Zeile hat gehört in den Fließtext.
Sie sollten für die Fließobjekt Caption keinen Satz bilden und 
daher auch keinen Punkt am Ende haben.
Die Caption ist eine Unterschrift und gehört unter das Fließobjekt.


%So kann man Bilder in den Text bauen mit wrapfigure
\begin{wrapfigure}{r}{6cm}
  \centering
  \includegraphics[width=4cm]{gnu}
  \caption{GNU-Logo~\cite{gnulogo,fal}}
  \label{fig:gnu}
\end{wrapfigure}
Es ist möglich, wenn auch nicht empfohlen, 
Bilder an den 
Rand einer Seite zu klatschen, wie wir das mit dem 
GNU-Logo in Abbildung~\ref{fig:gnu} gemacht haben. 
Das ganze ist ein netter Effekt für Graphiken, wie zum Beispiel ein
Logo, die nicht zum Verständnis des Texts gebraucht werden und wenig
Details aufweisen. 
Der Effekt sollte aber nicht überstrapaziert werden, 1--2 Mal 
je Abschlussarbeit sollte, wenn überhaupt, rei\-chen.
Außerdem funktioniert \verb|wrapfigure| nicht immer sehr stabil.



%So kann man Sachen fett machen
Bitte nehmen Sie \textbf{nie} JPG oder PNG für Vektorgrafiken, 
also Zeichnungen mit Linien oder anderen geometrischen Objekten,
sondern ausschließlich PDF.
Binden Sie also \textbf{nie} Vektorgraphiken verpixelt ein.


%einfach so mit pdf Endung dran machen
\begin{figure}[htb]
\centering
\includegraphics[width=.9\textwidth]{automata.pdf}
\caption{Automaten mit tikz~\cite{tikzautomata}}
\label{fig:tikz}
\end{figure}

%So kann man Code einbinden in 2 verschiedenen Zeichensätzen und auch aus einer Datei einlesen wie unten mit dem CPP Beispiel

Neben langen Listings sind natürlich kurze prägnante
Listings in Pseudocode (oder Python ;-) viel 
angenehmener, wie in Listing~\ref{code:ggt} der
effiziente GGT.

\begin{listing}[htbp]
\begin{lstlisting}
def ggt(x, y):
    while x != 0:
        x,y = y%x, x
    return y
\end{lstlisting}
\caption{ggT --- kurz und gut}
\label{code:ggt}
\end{listing}

Die Parameter für Listings sollte man für das ganze Dokument gleich 
lassen. 
Wenn man mal unbedingt wechseln will, dann ist das auch möglich,
wie zum Beispiel bei Listing~\ref{code:ggtjava}, das den ggT
in Java mit einem anderen Zeichensatz zeigt.

\begin{listing}[htbp]
\lstset{basicstyle=\sffamily, columns=[l]flexible, mathescape=true, showstringspaces=true, numbers=none, language=java}
\begin{lstlisting}
public static int ggt(int x, int y) {
    while (x != 0) {
        int h = x;
        x = y%x;
        y = h;
    }
    return y;
}
\end{lstlisting}
\caption{ggT --- Java}
\label{code:ggtjava}
\end{listing}

Der verwendete serifenlose Zeichensatz sieht vielleicht schöner 
aus, aber der variable Zeichenabstand kann bei Listing störend 
sein. Der in den Beispielen Listing~\ref{code:ggtaua} und 
Listing~\ref{code:ggt} verwendete Zeichensatz mit festem
Zeichenabstand ist für Quellcode meist zu bevorzugen.
Wir können natürlich auch C++-Quellcode setzen, bei Listings
\LaTeX\ in den Kommentaren verwenden und Listings aus
Dateien einlesen wie in~\ref{code:gcdcpp}.

\begin{listing}[htbp]
  \lstset{basicstyle=\ttfamily, columns=[l]flexible, mathescape=true, numbers=left, language=c++}
  \lstinputlisting{gcd.cpp}
\caption{gcd --- C/C++}
\label{code:gcdcpp}
\end{listing}


%So macht man Gänsefüßchen
Idealerweise verwenden Sie spätestens jede zweite Seite ein Bild. 
Ein Bild lockert auf und "`sagt mehr als tausend Worte"'.
Vermeiden Sie Aufzählungen.

%So macht man Mathe (falls ich das überhaupt brauche)


Eine Formel kann im Fließtext integriert sein, wie zum Beispiel 
$\sum_{i=1}^n i = \frac{n \cdot (n+1)}{2}$ oder separat
und referenzierbar gesetzt werden wie die Folgende:
\begin{equation} \label{eq:gauss}
  \sum_{i=1}^n i = \frac{n \cdot (n+1)}{2}
\end{equation}
Im \LaTeX-Quelltext werden beide Arten von Formeln gleich geschrieben 
aber anders gesetzt. 
Achten Sie zum Beispiel auf das Summenzeichen und die Positionen 
des Index. 
Die Gleichung~\ref{eq:gauss} ist natürlich vom Fließtext aus referenzierbar.
Man kann auch schreiben: 
(\ref{eq:gauss}) ist natürlich vom Fließtext aus referenzierbar.
Im Fließtext kann man auch gerne auf den Bruch verzichten und 
$\sum_{i=1}^n i = (n \cdot (n+1))/2$
schreiben, was meist etwas lesbarer ist.
Alternativ geht auch 
$\sum_{i=1}^n i = \mbox{\textonehalf} \cdot n \cdot (n+1)$.
Achten Sie bei Formeln darauf als Multiplikationszeichen 
$\cdot$ zu verwenden und nicht $*$. 
Ich kenne einen Kollegen, der ansonsten dadurch sehr erregt wird. 

Sie können viele Symbole, wie die griechischen Buchstaben 
$\alpha, \beta, \gamma, \ldots$;
logische Symbole wie $\forall, \exists, \not\exists, \wedge, \vee, \neg,
\Rightarrow, \Leftrightarrow$;
Mengensymbole wie $\in, \cup, \cap, \subseteq, \not\supset, 
\biguplus, \ldots$;
andere Symbole $\rightarrow, \sqsubseteq, \sim, 
\models, \vdash, \infty, \emptyset, \mathbb{N}, \mathbb{R}, \ldots$;
oder zusammengesetzte Gleichungen 
wie die Definition der 91er-Funktion~\cite{manna70}
verwenden.
\[
  f(x) = \left\{ \begin{array}{lll}
      x-10 & \mbox{gdw} & x>100 \\
      f(f(x+11)) & \multicolumn{2}{l}{\mbox{sonst}} 
    \end{array} \right.
\]

%So gehen dann Definitionen für Formeln

\begin{definition}
Sei $\varepsilon = 0$.
\end{definition}

\begin{satz}
Für alle positiven ganzen Zahlen $n$ gilt 
$\sum_{i=1}^n i = \frac{n \cdot (n+1)}{2}$ \enspace.
\end{satz}
\begin{beweis}
Vollständige Induktion:
\begin{itemize}
\item \emph{Induktionsanfang ($n=1$):} Es gilt  
\[
  \sum_{i=1}^1 i \, = \, 1 \, = \, \frac{1 \cdot (1+1)}{2}.
\]
\item \emph{Induktionsschritt ($n \rightarrow n+1)$:}
Es gelte die Induktionsvoraussetzung (IV):
\[
\sum_{i=1}^n i = \frac{n \cdot (n+1)}{2}
\]
Wir zeigen, dass dann auch 
\[
\sum_{i=1}^{n+1} i = \frac{(n+1) \cdot (n+2)}{2}
\]
gilt wie folgt:
\begin{eqnarray*}
  \sum_{i=1}^{n+1} i 
  & = & (\sum_{i=1}^{n} i) + (n+1)  \\
  & =_{\mbox{IV}} & \frac{n \cdot (n+1)}{2} + (n+1) \\
  & = & \frac{n \cdot (n+1)}{2} + \frac{2 \cdot (n+1)}{2} \\
  & = & \frac{n \cdot (n+1) + 2 \cdot (n+1)}{2} \\
  & = & \frac{(n+2) \cdot (n+1)}{2} 
\end{eqnarray*}
\end{itemize}
\qed
\end{beweis}
Sie müssen den Dreisatz 
\emph{Definition}, \emph{Satz} und \emph{Beweis} nicht verwenden, 
wenn Sie kein sehr formales Thema haben. 
Eine sehr formale Aufarbeitung von bekanntem Inhalt gefolgt 
von einem nicht so formalen eigenen Anteil sollte man 
meist vermeiden.

%So geht eine Website in den Text zu setzen
\url{http://www.csie.ntu.edu.tw/~cjlin/libsvm/}, 
sondern einen passenden publizierten Artikel zitieren.

%So kann man Befehle anzeigen
%eigene Zeile
\begin{verbatim}
$ bibtex thesis1
$ bibtex thesis2
\end{verbatim}
%in einer Zeile
\verb|bibtex|


\chapter{Zusammenfassung und Ausblick} \label{chap:fazit}

Abschließend kann man sagen, dass...

\newpage

% Listen wenn überhaupt ans Ende und nicht an den Anfang.
% Meist ist das aber unnötig.
% \listoffigures % Liste der Abbildungen 
% \listoftables % Liste der Tabellen
% \newpage

\bibliographystyle{plain} % Literaturverzeichnis
\begin{btSect}{thesis} % mit bibtopic Quellen trennen
\section*{Literaturverzeichnis}
\btPrintCited
\end{btSect}
\begin{btSect}{online}
\section*{Online-Quellen}
\btPrintCited
\end{btSect}
% dann mit "bibtex thesis1" und "bibtex thesis2" arbeiten

\end{document}
;;; Local Variables:
;;; ispell-local-dictionary: "de_DE-neu"
;;; End:
